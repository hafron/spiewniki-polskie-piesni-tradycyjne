\documentclass[a4paper,12pt]{article}
\usepackage{alltt}
\usepackage{fontspec}
\setmainfont{DejaVuSans}
\usepackage[polish]{babel}
\usepackage{pgfpages} 
\pgfpagesuselayout{2 on 1}[a4paper,landscape,border shrink=5mm]
\begin{document}

\section*{Anioł Gabryjel w poselstwie przychodzi}

\begin{alltt}
Anioł Gabriel w poselstwie przychodzi 
Do Nazaretu posłany
Zwiastując Pannie, że Syna porodzi 
Bez panieństwa odmiany.
Mówi, że poczniesz w czystym żywocie 
Boga-Człowieka w dwoi istocie,
Który będzie Świętym zwany.

Dziwi się Panna pomieszana zgoła,
Nowa myśl Jej rozum natęża.
"Jak to być może - rzecze do Anioła - 
Ponieważ nie znam męża?"
"Wierz, boś znalazła łaskę u Pana - 
Syna Boskiego Matkąś obrana,
Żebyś głowę starła węża."

"Oto ja Twoja służebnica Panie 
Na rozkaz Twój jestem gotowa.
Już wierzę - niech się wola Twoja stanie 
Według danego mi słowa.
Jeśli mój żywot ciasnć jest Boże, 
W sercu więc Ci dam przestrzeńsze łoże:
Miłość Cię moja wychowa.

O wielkie szczęście Maryja dla Ciebie, 
Które Cię dziś potkało,
Że Słowo Boskie w zbawienia potrzebie 
W Tobie ciałem się stało.
A więc o Matko Boska i nasza 
Racz już urodzić nam Mesyjasza,
Który z Ciebie dziś wziął ciało.
\end{alltt}
\newpage

\section*{Chodziła Maryja pod swojem domostwem}
\begin{alltt}
Chodziła Maryja pod swojem domostwem
I gadała, rozmawiała z Zbawicielem Boskiem.

Przyleciało do niej z podlesia ptaszątko
I gadało, rozmawiał: "Pordodzisz dzieciątko".

Maryja się tego wielce zawstydziła,
Swoje przenajdroższe oczki do ziemi spuściła.

Nie wstyź się Maryja, prze z Ducha Świętego
Coś poczęła to porodzisz Boga prawdziwego.

Na łączkę usiadłszy, kwiatuszków narwawszy
Wije wianki z maierzanki łzami się zalawszy.

Trzy wianki uwiła, tak se poślubiłą:
Pierwszy wianek Jezusowi na główkę włożyła.

W rugim sasa chodzi, bo tak się jej godzi, 
Trzeci na ołtarzu leży, bo tak się należy.

\end{alltt}
\newpage

\section*{Weselcie się ludzie}
\begin{alltt}
\end{alltt}
\newpage

\section*{Weselcie się ludzie}
Weselcie się ludzie.
\newpage

\end{document}
