\documentclass[a4paper,12pt]{article}
\usepackage{alltt}
\usepackage{fontspec}
\setmainfont{DejaVuSans}
\usepackage[polish]{babel}
\usepackage{pgfpages} 
\pgfpagesuselayout{2 on 1}[a4paper,landscape,border shrink=5mm]
\begin{document}


\section*{Anioł Gabryjel w poselstwie przychodzi}

\begin{alltt}
Anioł Gabriel w poselstwie przychodzi 
Do Nazaretu posłany
Zwiastując Pannie, że Syna porodzi 
Bez panieństwa odmiany.
Mówi, że poczniesz w czystym żywocie 
Boga-Człowieka w dwoi istocie,
Który będzie Świętym zwany.

Dziwi się Panna pomieszana zgoła,
Nowa myśl Jej rozum natęża.
"Jak to być może - rzecze do Anioła - 
Ponieważ nie znam męża?"
"Wierz, boś znalazła łaskę u Pana - 
Syna Boskiego Matkąś obrana,
Żebyś głowę starła węża."

"Oto ja Twoja służebnica Panie 
Na rozkaz Twój jestem gotowa.
Już wierzę - niech się wola Twoja stanie 
Według danego mi słowa.
Jeśli mój żywot ciasnć jest Boże, 
W sercu więc Ci dam przestrzeńsze łoże:
Miłość Cię moja wychowa.

O wielkie szczęście Maryja dla Ciebie, 
Które Cię dziś potkało,
Że Słowo Boskie w zbawienia potrzebie 
W Tobie ciałem się stało.
A więc o Matko Boska i nasza 
Racz już urodzić nam Mesyjasza,
Który z Ciebie dziś wziął ciało.
\end{alltt}


\section*{Chodziła Maryja pod swojem domostwem}
\begin{alltt}
Chodziła Maryja pod swojem domostwem
I gadała, rozmawiała z Zbawicielem Boskiem.

Przyleciało do niej z podlesia ptaszątko
I gadało, rozmawiał: "Pordodzisz dzieciątko".

Maryja się tego wielce zawstydziła,
Swoje przenajdroższe oczki do ziemi spuściła.

Nie wstyź się Maryja, prze z Ducha Świętego
Coś poczęła to porodzisz Boga prawdziwego.

Na łączkę usiadłszy, kwiatuszków narwawszy
Wije wianki z maierzanki łzami się zalawszy.

Trzy wianki uwiła, tak se poślubiłą:
Pierwszy wianek Jezusowi na główkę włożyła.

W rugim sasa chodzi, bo tak się jej godzi, 
Trzeci na ołtarzu leży, bo tak się należy.

\end{alltt}


\section*{Gdy pan Jezus we drzwi puka}
\begin{alltt}
Gdy Pan Jezus we drzwi puka, Oj lelu, lelu we drzwi puka
Matka Boska mu odmyka, Oj lelu, lelu, mu odmyka
Wejdźże Jezu w moje progi, Oj lelu, lelu w moje progi
Chociaż dom mój jest ubogi, Oj lelu, lelu jest ubogi
A na dachu gołąb grucha, Oj lelu, lelu gołąb grucha
I wygruchał bryłę złota, Oj lelu, lelu bryłę złota
Achłop oddał do złotnika, Oj lelu, lelu do złotnika
Złotnik ulał dwa kielichy, Oj lelu, lelu dwa kielichy
A któż z nich będzie pijał, Oj lelu, lelu będzie pijał
Sam Pan Jezus i Maryja, Oj lelu, lelu i Maryja.
\end{alltt}


\section*{Hej hej, lelija!}
\begin{alltt}
Hej hej, lelija, Panna Maryja!
Hej, porodziła Pana Jezusa
Panna Maryja.

Hej hej, lelija, Panna Maryja!
Tam między dwoma bydlątkoma
tam leży słoma barłożeczkoma,
tam porodziła Pana Jezusa(1)
Panna Maryja.

Hej hej, lelija, Panna Maryja!
W co powijała Pana Jezusa
Panna Maryja?
Hej hej, lelija, Panna Maryja!

W Najświętszej Panny pogłowniczek,
w świętego Józefa przypaśniczek,
w to powijała Pana Jezusa(1)
Panna Maryja.

Hej hej, lelija, Panna Maryja!
W czym kołysała Pana Jezusa
Panna Maryja.

Hej hej, lelija, Panna Maryja!
Tam między dwoma ołtarzykom
tam kolebeczka jest zawieszona,
tam kołysała Pana Jezusa(1)

Panna Maryja.
Hej hej, lelija, Panna Maryja!
Same się kościoły pootwierały,
bo się Pana Boga uradowały.
Panna Maryja.

Hej hej, lelija. Panna Maryja!
Same się świece pozapalały,
bo się Pana Boga uradowały.
Panna Maryja. [1], [2]

Objaśnienia:
(1) – Wiersz 4 śpiewa się na tę samą melodię co wiersze 2 i 3 (t. 5–8).
\end{alltt}


\section*{Jezu śliczny kwiecie}
\begin{alltt}
Jezu śliczny kwiecie, zjawiony na świecie:
A czemuż się w zimie rodzisz, * Ciężki mróz na się przynosisz,
                  Nie na ciepłem lecie, nie na ciepłem lecie?
Jezu niepojęty, czemu nie z panięty,
Nie w pałacuś jest złożony, * W lichej szopie narodzony,
                  I między bydlęty, i między bydlęty.
Niewinny Baranku, drżysz na gołym sianku:
Czem nie w złotej kolebeczce, * Nie na miękiej poduszeczce,
                  Niewinny Baranku, niewinny Baranku.
Śliczna jak lilija Panienka Maryja,
Cała piękna jako róża, * Nie szuka pańskiego łoża,
                  W żłobeczku powija, w żłobeczku powija.
Osiołek i z wołem stoją przed Nim społem,
Zagrzewają swego Pana, * Upadają na kolana,
                   Nisko biją czołem, nisko biją czołem.
Aniół z nieba budzi, najprzód prostych ludzi:
Pastuszkowie prędzej wstajcie, * W szopie Pana przywitajcie,
                   Co się dla was trudzi, co się dla was trudzi.
Pastuszkowie mali, prędko się zebrali,
To z muzyką, to z pieśniami, * To z różnemi ofiarami,
                   Panu cześć dawali, Panu cześć dawali.
Gwiazda asystuje i w drodze przodkuje,
Dokąd wschodu Monarchowie, * Jechać mają trzej Królowie,
                   Szopę pokazuje, szopę pokazuje.
Wielcy luminarze, księżyc z słońcem w parze,
Światłem swojem przyświecają, * Usługi Bogu oddają,
                   Światłości szafarze, światłości szafarze.
Zacny Opiekunie, Józefie piastunie,
Nie mogłeś znaleść gospody, * Jezusowi dla wygody,
                    I Najświętszej Pannie? I Najświętszej Pannie?
O szczęśliwa szopka, ubogiego chłopka,
W której Boga mego Ciało * Narodzone spoczywało,
                     Jest pokory próbka, jest pokory próbka.
W tem najświętszem Ciele, jest tajemnic wiele,
Tajemnic Boskich niemiara, * Których uczy święta wiara
                      W powszechnym Kościele, w powszaechnym Kościele.
O dobroci morze, niepojęty Boże!
Któż Ci godnie za te dary * Co sypiesz na nas bez miary,
                       Wydziękować może, wydziekować może.
O Jezu kochany, nam z nieba zesłany,
Przez Twe święte Narodzenie, * Daj szczęśliwe powodzenie,
                        Żywot pożądany, żywot pożądany.
\end{alltt}


\section*{Przylecieli tak śliczni Anieli}
\begin{alltt}
Przylecieli tak śliczni Anieli,
Wszyscy w bieli, złote piórka mieli:
Przynieśli nam wesołą nowinę,
Panna czysta zrodziła Dziecinę.
 
A zrodziwszy w pieluszki powiła,
A powiwszy na sianku złożyła:
Leży, leży Jezus malusieńki,
Leży, leży Jezus nagusieńki.

Zdjęła Panna swój rąbeczek z głowy,
Ściele w żłóbku Panu Jezusowi.
A nynajże drogie serce moje,
Bo cię kocham ponad życie swoje.

Pastuszkowie grajcie stwórcy swemu,
W tej stajence dla was zrodzonemu:
Proścież Pana by wam błogosławił,
Tu na ziemi, a w niebie postawił.
\end{alltt}


\section*{Stała nam się nowina miła}
\begin{alltt}
Stała nam się nowina miła
Panna Maryja syna powiła
Powiła go z wielkiem weselem
Będzie on naszym zbawicielem, zbawicielem

Król Heród się zafrasował
Wszystkie dziatki wyciąć kazał
Maryja się dowiedziała
Ze swym synkiem uciekała, uciekała

Gdy spotkała chłopka w polu orzący
Swoją pszeniczkę w ręku siejący
Siejże chłopku w imię moje
Jutro będziesz zbierał swoje, swoje

Nie powiadaj chłopku, że ja tędy szła
Maleńkie dziecię na ręku niosła.
Żydowie się dowiedzieli
Za Maryją pobieżeli, pobieżeli.

Gdy napadli chłopka w polu już żący
Swoją pszeniczkę w snopy wiążący
Powiedz że nam chłopku miły
Czyś nie widział ach Maryi, ach Maryi?

Widziałem ją ale łoni
Już Maryi nie zdogoni
Jeszcze się ta pszeniczka siała
Kiedy Maryja tędy bieżała, ach bieżała

Żydzi stanęli jakby trzcina
Bo ich moc boska bardzo zaćmiła
Maryja się dowiedziała
W ciemnym lasku nocowała, nocowała

Z lodu ognia ukrzesała
Pana Jezusa ugrzewała
Lulajże ach mocny Boże
Twój majestat ściele łoże, ściele łoże.
\end{alltt}


\section*{Śliczna Panienka}
\begin{alltt}
Śliczna Panienka, jako jutrzenka,
zrodziła Syna, dobra nowina.
W szopce ubogiej,
lubo mróz srogi,
w żłóbek włożyła Boskiego Syna.

Wiwat Pan Jezus,
wiwat Maryja!
Wiwat i Józef,
cna kompanija!

Coraz to dalej
szopka się wali,
Józef nieborak kijmi podpiera.
Wiatr zewsząd wieje,
nikt nie zagrzeje,
wicher do reszty strzechę obdziera,

Wiwat Pan Jezus...

Śliczna Matula
Dziecię utula,
karmi piersiami, szuka posłania.
Józef staruszek,
wziąwszy pieluszek,
zewsząd od wiatru dziury zasłania.

Wiwat Pan Jezus...

Dziecię się luli,
Matuchna tuli,
przestało przecie płakać po chwili:
wół z osłem stają,
parą chuchają,
klękając nisko, pokłon czynili.

Wiwat Pan Jezus...

Wtem Aniołowie
z nieba posłowie,
bieżąc do szopy, tak zaśpiewali:
Chwała bądź Bogu
na wysokości,
A ludziom pokój bądź w dobrej woli.

Wiwat Pan Jezus...

Pasterze wstają,
zbyt się lękają,
co to jest za głos, dalej czekają:
Święci Anieli,
wszyscy weseli,
To Dzieciąteczko im ogłaszają.

Wiwat Pan Jezus...

Wnet niebożęta,
biedne chłopięta,
spędzili z pola swoje bydlęta;
wraz się zmawiają,
co też wziąć mają,
Wielkiemu Panu ci pastuszęta.

Wiwat Pan Jezus...

Gdy się zmówili
i zgromadzili,
Wzięli dudeczki, przez drogę grali,
Wchodząc do szopy
jak proste chłopy,
co który przyniósł, Panu dawali.

Wiwat Pan Jezus...

Jedni pasterze
grali na lirze,
na fujareczkach drudzy śpiewali;
inni wesoło
tańczyli wkoło,
z wielkiej radości razem krzykali.

Wiwat Pan Jezus...

A gdy się byli
już ucieszyli,
Panu małemu dzięki składali,
że się narodził,
by oswobodził
lud od niewoli, wraz zawołali.

Wiwat Pan Jezus...

My Bogiem Ciebie
znamy na niebie,
Maryję Pannę za Matkę mamy:
teraz padamy
oraz błagamy,
nie opuszczaj nas, już Cię żegnamy.

Wiwat Pan Jezus...
\end{alltt}


\section*{Weselcie się ludzie}
\begin{alltt}
Weselcie się ludzie, już wam dobrze będzie,
Bóg zwalczył szatana, co zdradził Adama.

Ty piekielny smoku, koniec ci w tym roku,
Już ci łeb zdeptano, jako obiecano.

Jużci nic nie sprawisz, darmo się tu bawisz,
W rajuś nas z zazdrości, pozbawił radości.

Co Ewa straciła, Panna naprawiła,
Porodziła Syna, dziwna to nowina.

Anieli śpiewają, pokój ogłaszają,
Na ziemi wesele, że Bóg żyje w ciele.

Wszystko się zmieniło, jak nigdy nie było,
Wino rzeką ciecze, ciepło jakby w lecie.

Lwami drzewo wożą, niedźwiedziami orzą,
Zając z chartem siedzą, z jednej misy jedzą.

Liszka pasie kury, kot myszy i szczury,
Wilk owcom nie szkodzi, wespół z niemi chodzi.

Ptacy też wspomnieli, co przedtem umieli,
Gdyż fraktem śpiewają, na muzyce grają.

Skowronek dyszkantem, a sikora altem,
Wróbel zaś tenorem, gawron jest kantorem,

Żóraw organistą, a bocian lutnistą,
Sroka graa w cymbały, wrona zaś w regały,

A kaczor na flecie, gąsior na klarnecie:
Kos skrzypki szykuje, kruk smyczek smaruje,

Bąk dudy nadyma, sowa puzon trzyma,
Dudek w szałamaje, łabędź takty daje.

I drzewa też znają, opak owoc dają,
Jabłka na dębinie, gruszki na sośninie,

Na wierzbinie wiśnie, na iwinie trześnie,
Bez zakwitł figami, jesion rozenkami.

Na głogu brzoskiwnie, migdał na tarninir,
Miód płynie z kloniny, oliwa z brzeziny.

Gruda też grudniowa, jak pigułka zdrowa,
Śnieg i lód styczniowy, słodki jak cukrowy.

Słusznie się radować, a Bogu dziękować,
Iż przez narodzenie zmienił przyrodzenie.

Szatana zwojował, w piekle go przykował,
A człeka grzesznego, wziął za brata swego.
\end{alltt}


\end{document}
